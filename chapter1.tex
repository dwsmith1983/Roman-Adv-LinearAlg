\part{Basic Linear Algebra}

\chapter{Vector Spaces}

\begin{exercise}
\item
  Let \(V\) be a vector space over \(F\).
  Prove that \(0\mathbold{v} = \mathbold{0}\) and
  \(r\mathbold{0} = \mathbold{0}\) for all \(\mathbold{v}\in V\) and
  \(r\in F\).
  Describe the different \(0\)'s in these equations.
  Prove that if \(r\mathbold{v} = \mathbold{0}\), then \(r = 0\) or
  \(\mathbold{v} = \mathbold{0}\).
  Prove that \(r\mathbold{v} = \mathbold{v}\) implies that
  \(\mathbold{v} = \mathbold{0}\) or \(r = 1\).
  \par\smallskip
  With \(0\mathbold{v}\), we scalar multiplication; that is, \(0\in F\) and
  \(\mathbold{v}\in V\).
  Therefore, \(0\cdot v_i\) for \(i = 1,2,\ldots\) and where \(v_i\) is the
  \(i\)-th position of \(\mathbold{v}\).
  Now, \(0\cdot v_i = 0\) for \(i\) so \(0\mathbold{v}=\mathbold{0}\) where
  \(\mathbold{0}\) is the zero vector.
  In the case of \(r\mathbold{0}\) where \(r\in F\) and \(\mathbold{0}\in V\),
  we have for each \(i\)-th position of the zero vector a zero.
  Thus, \(r\cdot 0_i\Rightarrow\mathbold{0}\) for each \(i\) again leaving us
  with the zero vector.
  In the first problem, where we have \(0\mathbold{v}\), zero was a scalar in
  the field \(F\); however, the solution to \(0\mathbold{v} = \mathbold{0}\)
  was the zero vector in \(V\).
  In second problem, where we have \(r\mathbold{0}\), zero was a vector in \(V\) and the solution \(r\mathbold{0} = \mathbold{0}\) was also the zero
  vector in \(V\).
  \par\smallskip
  Suppose on the contrary that \(r\mathbold{v} = \mathbold{0}\), \(r\neq 0\),
  and \(\mathbold{v}\neq \mathbold{0}\).
  Since \(r\neq 0\), we can divide out by \(r\) so
  \(r\mathbold{v} = \mathbold{0}\iff \mathbold{v} = \mathbold{0}\).
  Thus, we have reached a contradiction; therefore, if
  \(r\mathbold{v} = \mathbold{0}\), then either \(r = 0\) or
  \(\mathbold{v} = \mathbold{0}\).
  \par\smallskip
  Let \(v_i\) be the \(i\)-th component of \(\mathbold{v}\).
  Then \(rv_i = v_i\) for all \(i\).
  Now, suppose on the contrary that \(r\mathbold{v} = \mathbold{0}\),
  \(r\neq 1\), and \(\mathbold{v}\neq \mathbold{0}\).
  Now, the equation \(rv_i = v_i\iff r = 1\) or \(v_i = 0\).
  Hence, we have reached a contradiction, and if
  \(r\mathbold{v} = \mathbold{v}\), then either \(r = 1\) or
  \(\mathbold{v} = \mathbold{0}\).
\item
  Prove theorem \(1.3:\) The set \(S(V)\) of all subspaces of a vector space
  \(V\) is a complete lattice under set inclusion, with smallest element
  \(\{0\}\), largest element \(V\), meet
  \[
    \glb\{S_i\colon i\in K\} = \bigcap_{i\in K}S_i
  \]
  and join
  \[
    \lub\{S_i\colon i\in K\} = \sum_{i\in K}S_i.
  \]
  We are given that the maximal and minimal elements of \(S(V)\) are \(V\) and
  \(\{0\}\), respectively.
  Therefore, we only need to show that each pair of elements has a meet and a
  join.
  If \(S_i\not\subset S_j\) for all \(i,j\in K\), then
  \(\bigcap_i S_i = \varnothing\) so the meet is the empty set, \(\{0\}\).
  Since \(\varnothing\in S_i\) trivial for all \(i\in K\), the set inclusion
  property is satisfied.
  For the join, we have \(\sum_iS_i = \bigcup_iS_i\) so the join is the union
  of the sets.
  Now is \(S_j\in\bigcap_iS_i\), then \(S_j\subset\bigcap_iS_i\) so the set
  inclusion property is satisfied.
  Suppose \(S_j\neq\varnothing\) and \(S_j\subset S_i\) for some \(i,j\in K\).
  Then \(\bigcap_iS_i = S_j\) for \(S_j\) the smallest set of the union; that
  is, \(S_j\subset S_1\), \(S_j\subset S_2\), and so on where either
  \(S_i\not\subset S_{i + 1}\) or \(S_i\subset S_{i + 1}\) for \(i\neq j\).
  Therefore, the meet is the smallest set \(S_j = \bigcap_iS_i\).
  Again, the join is simple the union of all the sets in \(\sum_iS_i\) which
  could be the maximal element \(V\).
  Thus, \(S(V)\) is a complete lattice.
\item
  \begin{exercise}[label = (\alph*)]
  \item
    Find an abelian group \(V\) and a field \(F\) for which \(V\) is a vector
    space over \(F\) in at least two different ways, that is, there are two
    different definitions of scalar multiplication making \(V\) a vector space
    over \(F\).
  \item
    Find a vector space \(V\) over \(F\) and a subset \(S\) of \(V\) that is
    \begin{enumerate*}[label = (\arabic*)]
    \item
      a subspace of \(V\) and
    \item
      a vector space using operations that differ from those of \(V\).
    \end{enumerate*}
  \end{exercise}
\item
  Suppose that \(V\) is a vector space with basis
  \(\mathcal{B} = \{b_i\colon i\in I\}\) and \(S\) is a subspace of \(V\).
  Let \(\{B_1,\ldots,B_k\}\) be a partition of \(\mathcal{B}\).
  Then is it true that
  \[
    S = \bigoplus_{i = 1}^k(S\cap\langle B_i\rangle)
  \]
  What if \(S\cap\langle B_i\rangle\neq\varnothing\) for all \(i\)?
\end{exercise}

%%% Local Variables:
%%% mode: latex
%%% TeX-master: t
%%% End:
