\part{Basic Linear Algebra}

\chapter{Vector Spaces}

\begin{exercise}
\item
  Let \(V\) be a vector space over \(F\).
  Prove that \(0v = 0\) and \(r0 = 0\) for all \(v\in V\) and \(r\in F\).
  Describe the different \(0\)'s in these equations.
  Prove that if \(rv = 0\), then \(r = 0\) or \(v = 0\).
  Prove that \(rv = v\) implies that \(v = 0\) or \(r = 1\).
  \par\smallskip
  With \(0v\), we scalar multiplication; that is, \(0\in F\) and \(v\in V\).
  Therefore, \(0\cdot v_i\) for \(i = 1,2,\ldots\) and where \(v_i\) is the
  \(i\)-th position of \(v\).
  Now, \(0\cdot v_i = 0\) for \(i\) so \(0v=\mathbold{0}\) where
  \(\mathbold{0}\) is the zero vector.
  In the case of \(r0\) where \(r\in F\) and \(0\in V\), we have for each
  \(i\)-th position of the zero vector a zero.
  Thus, \(r\cdot 0_i\Rightarrow\mathbold{0}\) for each \(i\) again leaving us
  with the zero vector.
  In the first problem, where we have \(0v\), zero was a scalar in the field
  \(F\); however, the solution to \(0v = 0\) was the zero vector in \(V\).
  In second problem, where we have \(r0\), zero was a vector in \(V\) and the
  solution \(r0 = 0\) was also the zero vector in \(V\).
  \par\smallskip
  Suppose on the contrary that \(rv = 0\), \(r\neq 0\), and \(v\neq 0\).
  Since \(r\neq 0\), we can divide out by \(r\) so \(rv = 0\iff v = 0\).
  Thus, we have reached a contradiction; therefore, if \(rv = 0\), then either
  \(r = 0\) or \(v = 0\).
\item
  Prove theorem \(1.3:\) The set \(S(V)\) of all subspaces of a vector space
  \(V\) is a complete lattice under set inclusion, with smallest element
  \(\{0\}\), largest element \(V\), meet
  \[
  \glb\{S_i\colon i\in K\} = \bigcap_{i\in K}S_i
  \]
  and join
  \[
  \lub\{S_i\colon i\in K\} = \sum_{i\in K}S_i.
  \]
\item
  \begin{exercise}[label = (\alph*)]
  \item
    Find an abelian group \(V\) and a field \(F\) for which \(V\) is a vector
    space over \(F\) in at least two different ways, that is, there are two
    different definitions of scalar multiplication making \(V\) a vector space
    over \(F\).
  \item
    Find a vector space \(V\) over \(F\) and a subset \(S\) of \(V\) that is
    \begin{enumerate*}[label = (\arabic*)]
    \item
      a subspace of \(V\) and
    \item
      a vector space using operations that differ from those of \(V\).
    \end{enumerate*}
  \end{exercise}
\item
  Suppose that \(V\) is a vector space with basis
  \(\mathcal{B} = \{b_i\colon i\in I\}\) and \(S\) is a subspace of \(V\).
  Let \(\{B_1,\ldots,B_k\}\) be a partition of \(\mathcal{B}\).
  Then is it true that
  \[
  S = \bigoplus_{i = 1}^k(S\cap\langle B_i\rangle)
  \]
  What if \(S\cap\langle B_i\rangle\neq\varnothing\) for all \(i\)?
\end{exercise}

%%% Local Variables:
%%% mode: latex
%%% TeX-master: t
%%% End:
